\section{Summary}

Krylov and multigrid methods based on domain decomposition techniques are powerful solvers for elliptical PDEs. Both are able to use coarse grids, which help to spread global information faster over the whole area. Especially for elliptical equations, this property is essential to achieve convergence rates independent of mesh number and grid width. An important criterion is to achieve a high degree of data locality. This is achieved by distributing the sub-area-oriented operations among the available processors and solving them independently from each other.

TOFIX: It represents a class of generalised multilevel domain decomposition/multigrid solvers that are highly specialised for scalar elliptic equations

TOFIX: To achieve optimal performance, solvers have to exhibit high arithmetic intensity and need to exploit every form of parallelism available in modern manycore CPUs. The computationally most expensive components of the solver are the repeated applications of the linear operator and the preconditioner.

TOFIX: To achieve good scalability, (global) communication obviously has to be minimised, and all locality requirements as outlined in the previous section apply.

TOFIX: Communication in parallel (HPC) compute clusters is much slower than computation.

 
TOFIX:
The default version of \scarc{} is based on a global, data-parallel CG-iteration with different block-preconditioning techniques (block-Jacobi, block-SSOR, block-FFT, block-Pardiso). This ansatz already shows a very     good numercial stability and accuracy as far as the domain decomposition is isotropic or moderately anisotropic. For more irregular decompositions it will be indispensable to implement the global, data-parallel MG-it    eration, especially with ADI-TRIGS-smoothing techniques, as described in %
 %\mcite{Kilian} \cite{Kilian2002}. 


TOFIX: Powerful solvers for elliptic problems that are able to meet the requirements concerning the numerical efficiency are multilevel domain decomposition (MLDD) and multigrid methods (MG). Both approaches employ coarse grids and are thus able to 'transport information faster through the domain'. For elliptic problems, this feature is mandatory for achieving good convergence rates that do not depend on the refinement level of the mesh. So, the question is how the two solution approaches can be parallelised. In multilevel domain decomposition methods, which are described in more detail in Section 2.2, the computational domain is subdivided into overlapping subdomains. Parts of the overall solution process are performed independently on these subdomains, such that MLDD methods can be naturally parallelised by distributing the subdomains and the corresponding operations to the available processors.


TOFIX: Unfortunately, not only new implementation techniques are necessary for massive efficiency enhancement, but also the mathematical ingredients that means the algorithms for the solution process and the discretizations have to be modified
Not data processing, but data moving is costly. Employ cache-oriented techniques based on data locality and pipelining. Exploit locally structured data.
